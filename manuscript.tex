\documentclass{article}
\usepackage[margin=1in]{geometry}
\usepackage{natbib}
\usepackage{graphicx}
\usepackage{amsmath}

\title{Example of htlatex}
\author{Thomas J. Leeper}

\begin{document}

\maketitle

\noindent This is a manuscript. Section \ref{sec:table} has a table. Section \ref{sec:figure} has a figure. Section \ref{sec:eq} has an equation. For demonstration purposes, here's a reference \citep{Grovesetal2006}.\footnote{This is a footnote.}

\subsection{A Table}\label{sec:table}

This section has a table. It's Table \ref{tab:table1}.

\begin{table}[ht]
\caption{An example table}\label{tab:table1}
\begin{tabular}{ll}
Header 1 & Header 2\\
Row 1 & Value 1 \\
Row 2 & Value 2 \\
\end{tabular}
\end{table}

\subsection{A Figure}\label{sec:figure}

This section has a figure. It's Figure \ref{fig:fig1}.

\begin{figure}[ht]
\caption{An example figure}\label{fig:fig1}
\includegraphics[height=2in]{figure.png}
\end{figure}

\subsection{An equation}\label{sec:eq}

This section has an in-line equation: $x^2 = 2^2 = 4$ and an equation environment:

\begin{equation}
a^2 = b^2 + c^2
\end{equation}

\noindent That's it.

\bibliographystyle{apsa-leeper}
\bibliography{references}

\end{document}
